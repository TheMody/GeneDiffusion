
In this work we presented the first full human genome diffusion models.
By creating synthetic human genomes we are able to address the problem of sharing highly sensitive data by replacing it with non sensitive synthetic data. We show that the synthetic data quality is sufficient to train on and achieve good results for ALS classification, as well as use a variety of other metrics to evaluate the quality of the generated data.

Further work is needed to extend this proof of concept to real world applications. The first step could be to improve the generated genomes, either by increasing the available training data for training the diffusion model or improving the diffusion model architecture. 

We look forward to the input of the scientific community and potential applications.

We publish all code at [anonymized for review].